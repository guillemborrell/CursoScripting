\documentclass{beamer}

\usepackage[utf8]{inputenc}
\usepackage[spanish]{babel}
\usepackage[T1]{fontenc}
\usepackage{graphicx}
\usepackage{tikz}
\usepackage{pgf}

% \usepackage{pgfpages}
% \pgfpagesuselayout{2 on 1}[a4paper,border shrink=5mm]

\renewcommand\shorthandsspanish{}
\noextrasspanish

\title{Python}
\author{
Guillem Borrell i Nogueras\\
\texttt{guillem@torroja.dmt.upm.es}
}

\begin{document}

\begin{frame}
\begin{center}
 \includegraphics[width=9cm]{files/header.png}\\
\begin{large}
\textbf{Informe de Actividad}
\end{large}\\
Laboratorio de Mecánica de Fluidos Computacional\\
ETSI Aeronáuticos. UPM\\
\vspace{0.5cm}
\textit{Guillem Borrell i Nogueras}\\
\textit{Javier Jiménez Sendín}\\
Barcelona, 10 de Junio de 2009
\end{center}

\end{frame}


\begin{frame}
\frametitle{Índice}
\begin{itemize}
\item Recursos y contrataciones
\item Otros frentes abiertos
\item Capa límite turbulenta
\begin{itemize}
\item Estado actual
\item Recursos de cálculo
\item Problemas con las nuevas arquitecturas
\item Plan de trabajo
\end{itemize}
\end{itemize}
\end{frame}

\begin{frame}
\frametitle{Recursos y contrataciones}
\begin{itemize}
\item Técnico multimedia (3 meses)
\item Ingeniero de sistemas (1 año)
\begin{itemize}
\item Gestión de la infraestructura del Laboratorio
\item MPI+Infiniband
\item OpenMP
\end{itemize}
\item Postdoc.
\begin{itemize}
\item Incorporada ayer
\item Capa límite turbulenta
\end{itemize}
\end{itemize}
\end{frame}


\begin{frame}
\frametitle{Otros frentes abiertos}
\begin{tikzpicture}
  \node[draw=blue, fill=blue!20, rounded corners] (a) {
    \begin{minipage}{6.5cm}
      \begin{itemize}
      \item Cell Processor
        \begin{itemize}
        \item Playstation Challenge
        \item ¿Opción de futuro?
        \end{itemize}
      \item OpenCL
        \begin{itemize}
        \item GPGPU
        \item Simulación de nubes de contaminante
        \end{itemize}
      \item Proyecto Piloto
        \begin{itemize}
        \item PIC
        \item Red Española de e-Ciencia
        \item Datos de simulaciones MN
        \item Almacenamiento Cloud
        \end{itemize}
      \end{itemize}
    \end{minipage}
  };
  \draw (5.2,-3.0) node[draw=blue, rounded corners] (cf) {
    \pgfimage[width=0.45\linewidth]{files/nube}
  };
\end{tikzpicture}
\end{frame}


\begin{frame}
\frametitle{Buscando una capa límite asintótica}
\framesubtitle{(la capa límite)}

\begin{itemize}
\item[2009] Simens, Jiménez, Hoyas, Mizuno. $Re_\theta = 2100$
  \begin{itemize}
  \item Mare Nostrum
  \end{itemize}
  \pause
\item[2009] Schlatter, Örlü, Li, Fransson, Johansson, Alfredsson,
  Henningson. $Re_\theta = 2500$
  \begin{itemize}
  \item Mayor $Re_\theta$ pero menor resolución.
  \end{itemize}
  \pause
\item[20??] $Re_\theta = 5500$
  \begin{itemize}
  \item $45 \times 10^9$ puntos (15 veces mayor)
  \item No hay ningún recurso en España con esta capacidad.
  \item Mare Nostrum en Nº60 Top500 (Junio 2009)
  \item 3 meses en 32000 cores.
  \end{itemize}
\end{itemize}

\pause

\begin{center}
  \begin{tikzpicture}
    \node[draw=red, fill=red!20, rounded corners] (cf)
    {Estamos obligados a buscar otros recursos.};
  \end{tikzpicture}
\end{center}

\end{frame}


\begin{frame}
\frametitle{Recursos disponibles}
\begin{itemize}
\item \textit{Intrepid}, $65 \times 10^6$ Horas
  \begin{itemize}
  \item Argonne National Laboratory (DoE)
  \item Blue Gene/P
  \item Texas A\&M
  \end{itemize}
\item DEISA, $20 \times 10^6$ Horas
  \begin{itemize}
  \item Forschungszentrum Jülich 
  \item Blue Gene/P
  \end{itemize}
\item Ottawa University
  \begin{itemize}
  \item Blue Gene/P
  \end{itemize}
\item Centro de Supercomputación de Castilla y León
  \begin{itemize}
  \item Intel Xeon \textbf{5450}
  \end{itemize}
\end{itemize}
\end{frame}

\begin{frame}
\frametitle{Blue Gene/P}
\begin{itemize}
\item Única opción a corto-medio plazo
\item Otra arquitectura más
\item ¿Adecuada para Turbulencia?
\begin{itemize}
\item FSB estrecho ¿?
\item 512 MB RAM / Core
\end{itemize}
\end{itemize}

\pause
\begin{center}
  \begin{tikzpicture}
    \node[draw=red, fill=red!20, rounded corners] (cf)
    {Código MPI, un proceso por core.};
  \end{tikzpicture}
\end{center}


\end{frame}


\begin{frame}
\frametitle{problemas de memoria}
\begin{tikzpicture}
\node[draw=blue, rounded corners] (cf) {
  \pgfimage[width=0.9\linewidth]{files/dominio}
};
\end{tikzpicture}

\end{frame}

\begin{frame}
\frametitle{problemas de memoria}
\begin{tikzpicture}
\node[draw=blue, rounded corners] (cf) {
  \pgfimage[width=0.9\linewidth]{files/planosespectral}
};
\end{tikzpicture}

\end{frame}


\begin{frame}
\frametitle{problemas de memoria}
\begin{tikzpicture}
\node[draw=blue, rounded corners] (cf) {
  \pgfimage[width=0.9\linewidth]{files/plano}
};
\end{tikzpicture}

\end{frame}



\begin{frame}
\frametitle{Paralelismo y concurrencia en el Blue Gene/P}
\begin{figure}
\label{fig:qtoctave-secciones}
  \centering
  \pgfimage[width=0.8\linewidth]{files/multicore}
  \caption{Tiempos de comunicación en función del tamaño de mensaje
    para 2 y 4 cores.}
\end{figure}

\end{frame}


\begin{frame}
\frametitle{Paralelismo y concurrencia en el Blue Gene/P}
\begin{figure}
\label{fig:qtoctave-secciones}
  \centering
  \pgfimage[width=0.8\linewidth]{files/latencia}
  \caption{Escalado del tiempo de comunicación en función del tamaño de mensaje
    para 1, 2 y 4 cores.}
\end{figure}

\end{frame}


\begin{frame}
\frametitle{problemas de memoria}
\begin{tikzpicture}
\node[draw=blue, rounded corners] (cf) {
  \pgfimage[width=0.9\linewidth]{files/esquemahilos}
};
\end{tikzpicture}
\begin{center}
\begin{tikzpicture}
  \node[draw=red, fill=red!20, rounded corners] (cf)
  {MPI + OpenMP};
\end{tikzpicture}
\end{center}
\end{frame}

\begin{frame}
\frametitle{Paralelismo y concurrencia en el Blue Gene/P}
\begin{figure}
\label{fig:qtoctave-secciones}
  \centering
  \pgfimage[width=0.8\linewidth]{files/latenciaopenmp}
  \caption{Escalado del tiempo de comunicación en función del tamaño de mensaje
    para 1, 2 y 4 cores.}
\end{figure}

\end{frame}


\begin{frame}
\frametitle{Conclusiones}
\begin{itemize}
\item En 1 año $Re_\theta = 5500$. Capa límite asintótica
\begin{itemize}
\item ¡Queremos los datos! Si es necesario que vengan a Barcelona a
  usarlos.
\item No nos fiamos de los suecos.
\item $\sim$ 100 TBi.  ¿Traerlos de EEUU?
\end{itemize}
\item Blue Gene/P
\begin{itemize}
\item MPI + OpenMP
\end{itemize}
\end{itemize}
\end{frame}

\end{document}
