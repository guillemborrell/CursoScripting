\documentclass[a4paper,10pt]{article}

\usepackage[utf8]{inputenc}
\usepackage{amsfonts}
\usepackage{amsmath}
\usepackage[spanish]{babel}
\usepackage[T1]{fontenc}
\usepackage{geometry}
\usepackage{graphicx}
\usepackage{mathptmx}
\usepackage{hyperref}
\usepackage{tikz}
\usepackage{listings}

\lstset{language=Python,
  backgroundcolor=\color{black!20},
  numbers=left,
  extendedchars=true,
  inputencoding=utf8,
  basicstyle=\small\ttfamily,
  numberstyle=\tiny,
  keywordstyle=\color{blue},
  showstringspaces=false,
  showspaces=false}


\renewcommand\shorthandsspanish{}
\noextrasspanish

\author{Guillem Borrell Nogueras}
\title{Python como entorno de desarrollo científico.\\ Segunda}


\begin{document}
\maketitle

\section{Orientación a objetos}

La orientación a objetos es un nuevo paradigma para la implementación
de algoritmos ya no tan nuevo.  Es una estructura que puede contener a
la vez variables y métodos propios y que dispone de ciertas
propiedades como la herencia y el polimorfismo.

\defverbatim[colored]\testcode{
\begin{lstlisting}

\end{lstlisting} 
}

\section{Numpy}

\section{Scipy}

\section{SAGE}

\section{Técnicas y herramientas para eliminar cuellos de botella}

\section{Supercomputación con Python}

\appendix

\section{Python 3.x}

\emph{Esta sección deberá eliminarse en cuando la siguiente versión de
Python, la 3.x se haya consolidado}

Python es en la actualidad un lenguaje en plena transición a una nueva
versión.  El gran objetivo de Python 3.x, anteriormente conocido como
Python 3000, es eliminar ciertos errores cometidos en los primeros
estadios del desarrollo del lenguaje y que se han mantenido en él por
la voluntad de no romper todo el código escrito. El objetivo
secundario es conseguir ahondar en la propia filosofía de Python como
lenguaje sencillo, consistente y corto.

A partir de la llegada de Python 3.x el código que ejecute en Python
2.x debe escribirse teniendo en cuenta muy seriamente los avisos del
intérprete en tiempo de ejecución.  En ellos se detallan problemas que
pueden aparecer en un futuro proceso de migración.

Es también una demostración de valentía.  Python es ya un lenguaje
maduro y este cambio puede romper infinidad de código que hoy funciona
sin ningún problema.  Lejos de ser algo que concierne sólo a unos
pocos desarrolladores sitios como Google, YouTube, NASA y miles de
empresas, sitios o proyectos de software tendrán que dedicar algo de
tiempo ---esperemos que no mucho--- al cambio de versión.

\section{Cómo instalar Python}
\label{sec:instalar}

CPython es el nombre que recibe habitualmente la implementación
estándar en C y la más difundida. Ha sido portada a una veintena de
sistemas operativos desde Windows a Symbian.

Python se encuentra ya instalado en todas las distribuciones populares
de GNU/Linux porque es un lenguaje de uso común para tareas de
configuración y administración del sistema.  Para instalar módulos
adicionales es recomendable utilizar el sistema de paquetes del propio
sistema operativo para que las actualizaciones se produzcan
automáticamente.

En la página web \url{http://python.org/}


\begin{thebibliography}{99}

\bibitem[TUT]{TUT}
  Python Tutorial; Guido van Rossum, Python Software Foundation;
  \newblock \url{http://docs.python.org/tut/tut.html}

\bibitem[REF]{REF}
  Python Reference Manual; Guido van Rossum, Python Software Foundation;
  \newblock \url{http://docs.python.org/ref/ref.html}

\bibitem[STB]{STB}
  Python Library Reference; Guido van Rossum, Python Software Foundation;
  \newblock \url{http://docs.python.org/lib/lib.html}

\bibitem[OCT]{OCT}
  Introducción Informal a Matlab y Octave; Guillem Borrell i Nogueras;
  \newblock \url{http://iimyo.forja.rediris.es}. ISBN: 978-8-4691-3626-3

\end{thebibliography}

\end{document}
