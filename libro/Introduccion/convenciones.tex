\section{Convenciones tipográficas}

A continuación se listan algunas de las convenciones tipográficas con
significado:

% * Cada vez que se exponga código literal que puede ser cortado y
%   pegado se utilizará una fuente equiespaciada.  Esto es un ejemplo::
% 
%     #include <stdout.h>
% 
%     int main(){
%       printf("Hello, world!);
%     }
% 
% * Durante los ejemplos es posible que haya que comunicarse con el
%   sistema mediante la consola.  Las sesiones de consola en UNIX se
%   expresarán del siguiente modo::
% 
%     U> cat /proc/version
%     Linux version 2.6.20-gentoo-r8 (root@peret)
%     (gcc version 4.1.2 (Gentoo 4.1.2)) #1 SMP 
%     Fri May 25 11:27:27 CEST 2007
%     U>
% 
%   Y las sesiones en consola windows como sigue::
% 
%     W> cd c:\
%     W> dir
%       El volumen de la unidad C no tiene etiqueta.
%       El número de serie del volumen es: 5C58-8AD2
% 
%       Directorio de C:\
% 
%     20/07/2007  12:06    <DIR>          Archivos de programa
%     21/12/2006  01:11                 0 AUTOEXEC.BAT
%     21/12/2006  01:11                 0 CONFIG.SYS
%     08/01/2007  20:17    <DIR>          cygwin
%     19/07/2007  14:01    <DIR>          Documents and Settings
%     19/07/2007  13:35    <DIR>          MinGW
%     21/12/2006  01:36    <DIR>          MyWorks
%     29/12/2006  13:40    <DIR>          Program Files
%     19/07/2007  13:08    <DIR>          Python25
%     26/03/2007  15:30         3.117.086 tec
%     19/07/2007  13:13    <DIR>          WINDOWS
%                3 archivos      3.117.086 bytes
%                8 dirs  142.115.049.472 bytes libres
% 
% * Además de la consola del sistema operativo, los intérpretes
%   utilizados también siguen una notación.  Python puede distinguirse
%   gracias a sus tres signos `>`::
% 
%     >>> print 'Hello, world!'
%     Hello, world!
%     >>> for i in range(2):print i
%     ...
%     0
%     1
% 
% 
%   Y para octave se utilizará la expresión de consola típica de
%   Matlab::
% 
%     >> a=[1,2,3];
%     >> a'
%     ans =
% 
%        1
%        2
%        3
% 
