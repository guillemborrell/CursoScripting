\chapter{Introducción}

\section{Objetivos de este libro}

Este es un proyecto ambicioso.  Nace de la hipótesis que los lenguajes
interpretados modernos pueden aportar grandes ventajas al desarrollo
de aplicaciones científicas sea cual sea su fin; que sus virtudes
pueden llegar a superar con creces sus defectos.  Los lenguajes
interpretados ayudan a programar más deprisa, con menos errores,
necesitan menos conocimientos de hardware... Sin embargo la llave de
vuelta de este libro es quizás el punto fuerte que menos se tiene en
cuenta: los lenguajes interpretados pueden extenderse mediante el uso
de lenguajes compilados.

Esto abre todo un abanico de posibilidades siendo la más importante el
uso de un lenguaje interpretado como `pegamento` para rutinas escritas
en C, C++, Fortran...  Ya no lo vemos como una herramienta de
desarrollo en el sentido clásico sino como un entorno de desarrollo e
integración.

Ya no es necesario que todos los miembros del equipo escriban en el
mismo lenguaje, cada uno puede utilizar el que más domine.  Ya no es
necesario reimplementar código ya existente simplemente porque no se
ha escrito en el lenguaje de la aplicación.  Son ejemplos de las
ventajas de utilizar un lenguaje pegamento o `binding language` como
núcleo de nuestros proyectos.

Se han escogido dos lenguajes auxiliares para ilustrar el proceso,
Python y Octave.  Pero esto no es en absoluto una monografía sobre
estos dos lenguajes.  También se utilizará C y Fortran pero no se
hablará sobre sus particularidades.

El objetivo principal de este texto es analizar una nueva manera de
escribir software para simulación.
