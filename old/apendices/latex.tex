\chapter{Pequeña introducción a \TeX{}\index{TeX} y
  \LaTeX{}\index{LaTeX}}

\TeX{} es un lenguaje de programación de textos y \LaTeX{} una
colección de macros que simplifican enormemente su uso. \LaTeX{} es un
lenguaje dedicado a tipografía matemática con cierto parecido a HTML.
Durante muchos años ha sido muy utilizado y sigue siendo el estándar
para la edición de artículos científicos. Es sido tan popular dentro
de su ámbito que sus códigos para definir caracteres se han extendido
a otros lenguajes y a los editores de fórmulas.

\LaTeX{} es capaz de preparar documentos con un acabado profesional
con cualquer PC como única herramienta. El aprendizaje no es sencillo
debido a la gran influencia que actualmente tienen los procesadores de
texto y el WYSIWYG%
\footnote{What You See Is What You Get%
}. Cuando nos dicen que para obtener un documento \LaTeX{} hay que
compilar un archivo de documento y que la salida es en formato DVI
probablemente tengamos un escalofrío. Lo que pasa es que hemos ganado
potencia a base de bajar el nivel de la aplicación. Si nos miramos un
procesador de texto detenidamente veremos que en realidad no estamos
escribiendo lo que sale en pantalla; en realidad el procesador escribe
un documento en formato XML que tiene muy poco que ver con el
resultado en pantalla. Escribir en \LaTeX{} sería como si nos
pusieramos a teclear directamente XML y pasáramos del procesador de
textos.

Cuanto más potente es una herramienta más esfuerzo requiere dominarla,
aún así para hacer los primeros pasos con el \LaTeX{} no se requieren
más de unas horas. Para empezar es importante tener un libro o una
guía. Es un lenguaje que se basa en comandos con lo que tener un
listado convenientemente explicado es esencial. Para encontrar algún
tutorial se puede acceder a la página del grupo de usuarios de
\LaTeX{} de España, \url{http://filemon.mecanica.upm.es/CervanTeX/}.

Otra opción es optar por algún procesador de textos \LaTeX{}, una
especie de híbrido entre procesador de textos y escritura \LaTeX{}
directa. No hay mucha variedad, suelen llamarse procesadores WYSIWYM%
\footnote{Whay You See Is What You Mean%
} porque en pantalla no se ve el output definitivo, sigue siendo
necesario compilar el documento. Los dos más conocido son LyX y
Scientific Word.

Este documento ha sido preparado con LyX. Es un gran procesador de
textos libre y gratuito y completamente multiplataforma; esto es,
tiene versiones para todos los sistemas operativos mayoritarios. Es um
programa de gran calidad, con mucha documentación adjunta y una
comunidad muy activa. Merece la pena probarlo.

Scientific Word, al igual que su hermano mayor Scientific Workplace,
es un entorno parecido a LyX pero más orientado hacia Windows.  Tiene
cierta compatibilidad con Microsoft Office y soporta mejor los
formatos de este sistema operativo. La diferencia entre Scientific
Word y Scientific Workplace es que el segundo cuenta además con un
sistema de cálculo simbólico embebido. Sólo existe versión para
Windows.

Es muy importante tener nociones de la sintaxis de \LaTeX{} porque
Matlab soporta sus símbolos en los gráficos y porque la mayoría de los
editores de ecuaciones basan el lenguaje de su intérprete en las
mismas palabras clave.


\section{Tabla con algunos caracteres \TeX{}.}

\begin{center}

\begin{tabular}{|c|c|c|c|c|c|}
\hline 
código&
símbolo&
código&
símbolo&
código&
símbolo\tabularnewline
\hline
\hline 
\textbackslash{}alpha&
$\alpha$&
\textbackslash{}simeq&
$\simeq$&
\textbackslash{}sim&
$\sim$\tabularnewline
\hline 
\textbackslash{}beta&
$\beta$&
\textbackslash{}upsilon&
$\upsilon$&
\textbackslash{}leq&
$\leq$\tabularnewline
\hline 
\textbackslash{}gamma&
$\gamma$&
\textbackslash{}varphi&
$\varphi$&
\textbackslash{}infty&
$\infty$\tabularnewline
\hline 
\textbackslash{}delta&
$\delta$&
\textbackslash{}chi&
$\chi$&
\textbackslash{}clubsuit&
$\clubsuit$\tabularnewline
\hline 
\textbackslash{}epsilon&
$\epsilon$&
\textbackslash{}psi&
$\psi$&
\textbackslash{}diamondsuit&
$\diamondsuit$\tabularnewline
\hline 
\textbackslash{}zeta&
$\zeta$&
\textbackslash{}omega&
$\omega$&
\textbackslash{}heartsuit&
$\heartsuit$\tabularnewline
\hline 
\textbackslash{}eta&
$\eta$&
\textbackslash{}Gamma&
$\Gamma$&
\textbackslash{}spadesuit&
$\spadesuit$\tabularnewline
\hline 
\textbackslash{}theta&
$\theta$&
\textbackslash{}Delta&
$\Delta$&
\textbackslash{}leftrightarrow&
$\leftrightarrow$\tabularnewline
\hline 
\textbackslash{}vartheta&
$\vartheta$&
\textbackslash{}Theta&
$\Theta$&
\textbackslash{}leftarrow&
$\leftarrow$\tabularnewline
\hline 
\textbackslash{}iota&
$\iota$&
\textbackslash{}Lambda&
$\Lambda$&
\textbackslash{}uparrow&
$\uparrow$\tabularnewline
\hline 
\textbackslash{}kappa&
$\kappa$&
\textbackslash{}Xi&
$\Xi$&
\textbackslash{}rightarrow&
$\rightarrow$\tabularnewline
\hline 
\textbackslash{}lambda&
$\lambda$&
\textbackslash{}Pi&
$\Pi$&
\textbackslash{}downarrow&
$\downarrow$\tabularnewline
\hline 
\textbackslash{}mu&
$\mu$&
\textbackslash{}Sigma&
$\Sigma$&
\textbackslash{}circ&
$\circ$\tabularnewline
\hline 
\textbackslash{}nu&
$\nu$&
\textbackslash{}Upsilon&
$\Upsilon$&
\textbackslash{}pm&
$\pm$\tabularnewline
\hline 
\textbackslash{}xi&
$\xi$&
\textbackslash{}Phi&
$\Phi$&
\textbackslash{}geq&
$\geq$\tabularnewline
\hline 
\textbackslash{}phi&
$\phi$&
\textbackslash{}Psi&
$\Psi$&
\textbackslash{}propto&
$\propto$\tabularnewline
\hline 
\textbackslash{}rho&
$\rho$&
\textbackslash{}Omega&
$\Omega$&
\textbackslash{}partial&
$\partial$\tabularnewline
\hline 
\textbackslash{}sigma&
$\sigma$&
\textbackslash{}forall&
$\forall$&
\textbackslash{}bullet&
$\bullet$\tabularnewline
\hline 
\textbackslash{}varsigma&
$\varsigma$&
\textbackslash{}exists&
$\exists$&
\textbackslash{}div&
$\div$\tabularnewline
\hline 
\textbackslash{}tau&
$\tau$&
\textbackslash{}ni&
$\ni$&
\textbackslash{}neq&
$\neq$\tabularnewline
\hline 
\textbackslash{}equiv&
$\equiv$&
\textbackslash{}rightleftharpoons&
$\rightleftharpoons$&
\textbackslash{}mp&
$\mp$\tabularnewline
\hline 
\textbackslash{}Im&
$\Im$&
\textbackslash{}approx&
$\approx$&
\textbackslash{}hbar&
$\hbar$\tabularnewline
\hline 
\textbackslash{}otimes&
$\otimes$&
\textbackslash{}Re&
$\Re$&
\textbackslash{}oint&
$\oint$\tabularnewline
\hline 
\textbackslash{}cap&
$\cap$&
\textbackslash{}oplus&
$\oplus$&
\textbackslash{}supseteq&
$\supseteq$\tabularnewline
\hline 
\textbackslash{}supset&
$\supset$&
\textbackslash{}cup&
$\cup$&
\textbackslash{}subset&
$\subset$\tabularnewline
\hline 
\textbackslash{}int&
$\int$&
\textbackslash{}subseteq&
$\subseteq$&
\textbackslash{}nabla&
$\nabla$\tabularnewline
\hline 
\textbackslash{}rfloor&
$\rfloor$&
\textbackslash{}in&
$\in$&
\textbackslash{}cdots&
$\cdots$\tabularnewline
\hline 
\textbackslash{}lfloor&
$\lfloor$&
\textbackslash{}lceil&
$\lceil$&
\textbackslash{}ldots&
$\ldots$\tabularnewline
\hline 
\textbackslash{}perp&
$\perp$&
\textbackslash{}cdot&
$\cdot$&
\textbackslash{}prime&
$\prime$\tabularnewline
\hline 
\textbackslash{}wedge&
$\wedge$&
\textbackslash{}neg&
$\neg$&
\textbackslash{}top&
$\top$\tabularnewline
\hline 
\textbackslash{}rceil&
$\rceil$&
\textbackslash{}times&
$\times$&
\textbackslash{}mid&
$\mid$\tabularnewline
\hline 
\textbackslash{}vee&
$\vee$&
\textbackslash{}varpi&
$\varpi$&
\textbackslash{}sum&
$\sum$\tabularnewline
\hline 
\textbackslash{}langle&
$\langle$&
\textbackslash{}rangle&
$\rangle$&
\textbackslash{}iint&
$\iint$\tabularnewline
\hline
\end{tabular}

\end{center}

Además de estos caracteres es importante conocer algunas convenciones
de la escritura de fórmulas. Antes de cada clave se puede aplicar
un atributo para modificar el tipo de letra, por ejemplo el subíndice
se denota con \texttt{\_} y el superíndice con \texttt{\textasciicircum{}},
la cursiva con \texttt{\textbackslash{}it} y la negrita con 
\texttt{\textbackslash{}bf}.
Para escribir la fórmula siguiente:
$$\nabla_{j}\cdot\mathbf{A}=\sum_{i}\partial_{i}\mathbf{A}_{ij}$$
haremos
\begin{verbatim}
\nabla_j \cdot \bf A = \sum_j \partial_j \bf A_{ij}
\end{verbatim}

