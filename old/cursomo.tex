%%%%%%%%%%%%%%%%%%%%%%%%%%%%%%%%%%%%%%%%%%%%%%%%%%%%%%%%%%%%%%%%
%                                                              %
%   Introducción informal a Matlab y Octave                    %
%   Archivo de fuente de texto LaTeX                           %
%   Guillem Borrell Nogueras                                   %
%   Licencia :                                                 %
%     Creative Commons Public License                          %
%     Reconocimiento-NoComercial-SinObraDerivada 2.5 España    %
%   2005-2006                                                  %
%                                                              %
%   Builds with both LaTeX and pdfLaTeX                        %
%   Remember to run makeindex                                  %
%%%%%%%%%%%%%%%%%%%%%%%%%%%%%%%%%%%%%%%%%%%%%%%%%%%%%%%%%%%%%%%%

\documentclass[10pt,fleqn,a4]{book}
%   Makeindex support
\usepackage{makeidx}
\makeindex
%   Usual document characteristics
\usepackage[spanish]{babel}
\usepackage[T1]{fontenc}
\usepackage[utf8]{inputenc}
%   Page formatting
\usepackage{geometry}
\geometry{
  verbose,
  a4paper,
  tmargin=2cm,
  bmargin=2cm,
  lmargin=2.5cm,
  rmargin=1cm,
  headheight=0.5cm,
  headsep=0.5cm,
  footskip=1cm
}
%   I do not know what this is for
\setcounter{secnumdepth}{3}
\setcounter{tocdepth}{3}
%   Math packages
\usepackage{array}
\usepackage{float}
%   Several font packages
%   AMSmath and AMSsymbol packages may conflict with babel
%     just console garbage, press <Enter> and go
\usepackage{amsmath}
\usepackage{amssymb}
\usepackage{marvosym}
%   Creative common licenses symbols.  Requires cclicenes.sty
\usepackage{cclicenses}
%   Use roman fonts for text and formulas.  Verbatim text is
%   still computer modern.
%\usepackage{mathptmx}
%   Graphics support
\usepackage{graphicx}
%   Fancyhdr and fncychap styles.  Requires external file
%     fncyheader.sty
\usepackage{fancyhdr}

\pagestyle{fancy}
\fancyhead{}
\fancyfoot{}
\fancyhead[RE]{\slshape \rightmark}
\fancyhead[LO]{\slshape \leftmark}
\fancyhead[LE,RO]{\slshape \thepage}
\fancyfoot[RO]{\slshape{Introducción informal a Matlab y Octave}}
\renewcommand{\headrulewidth}{0.4pt}
\renewcommand{\footrulewidth}{0.4pt}
\usepackage[Conny]{fncychap}

%   Hyperref package to automatically generate hyperlinks
\usepackage{hyperref}
%\usepackage{indentverbatim}


%%%%%%%%%%%%%%%%%%%%%%%%%%%%%%%%%%%%%%%%%%%%
%                                          %
%                                          %
%      %%%Introducción informal a%%%       %
%      %%%Matlab y Octave        %%%       % 
%                                          %
%                                          %
%%%%%%%%%%%%%%%%%%%%%%%%%%%%%%%%%%%%%%%%%%%%

\begin{document}

\title{\begin{Huge}Introducción informal a Matlab y Octave\end{Huge}}


\author{Guillem Borrell i Nogueras\\
  \href{mailto:guillemborrell@gmail.com}
  {\texttt{guillemborrell@gmail.com}}\\
  \texttt{\url{http://torroja.dmt.upm.es:9673/Guillem\_Site/}}\\
}

\maketitle

%%%%%%%%%%%%%%%%%%%%%%%%%%%%%%%%%%%%%%%%%%%%%%%%%%%%%%%
\chapter*{Prólogo}
%%%%%%%%%%%%%%%%%%%%%%%%%%%%%%%%%%%%%%%%%%%%%%%%%%%%%%%

Hay muchos libros de Matlab, algunos muy buenos, pero en ninguno es
tratado como un lenguaje de programación. El enfoque habitual es
pensar en Matlab como programa, como un entorno de desarrollo
completo. No se habla nunca del intérprete Octave ni a las ventajas y
defectos respecto a otros lenguajes de programación. No son libros,
son manuales.  Creo que es muy importante aplicar el sentido crítico a
cualquier herramienta y todo lo editado hasta a hora no es de gran
ayuda. Octave es un programa magnífico, lo he usado durante años. No
llega a la magnitud de Matlab pero debe ser tenido en cuenta. En este
libro los dos intérpretes se analizan en igualdad de condiciones y se
apunta cuál puede ser la mejor opción en cada caso.

Estos apuntes empezaron como material adicional mal escrito para un
curso de seis horas; con tiempo y dedicación han crecido hasta lo que
son ahora. Escribir sobre un lenguaje de programación es largo,
difícil y laborioso; nunca sabes si el lector va entender los
conceptos que plasmas sobre el papel. Esto requiere el esfuerzo extra
de reducir las ideas a lo más básico. Es una experiencia gratificante,
sobre todo cuando uno mismo tiene que reformular conceptos que ya
creía asimilados. Uno aprende a escribir, a explicarse y a tener
paciencia.  Es un curso informal, pretende ser cercano y ameno incluso
cuando se tratan conceptos complejos o abstractos.

Este libro es libre y abierto; quería que fuera así desde un
principio.  Todo el que quiera participar en él puede hacerlo sin
ninguna restricción.  Su única finalidad es ayudar a los demás. Espero
que quien lo sostenga en sus manos aprecie esta pequeña muestra de
altruismo y decida colaborar; estaré siempre abierto a sugerencias y
correcciones. Incluso si alguien propone una reescritura o la
inclusión de un capítulo no tengo ningún reparo en otorgarle la
coautoría.\\
\\
\begin{flushright} Guillem Borrell i Nogueras \\
  Calella, 13 de Agosto de 2005

\end{flushright}

\pagebreak

Este documento está publicado según la siguiente licencia:
\begin{center}
\begin{Huge}
$\cc$ \quad Creative Commons\\
\end{Huge}
\textbf{Reconocimiento-NoComercial-SinObraDerivada 2.5 Espa\~na}\\
\end{center}


Copyright $\cc$ $\byncnd$ Guillem Borrell i Nogueras {}``Some rights
reserved''.\\


El documento está publicado bajo una licencia que no permite el
trabajo derivado. Sin embargo este documento puede publicarse bajo
otras licencias del tipo {}``algunos derechos reservados'' para
permitir la modificación libre del texto. Para obtener el texto sujeto
a una licencia alternativa es necesario ponerse en contacto con el
autor.\\


Matlab${\textregistered}$ y MathWorks${\textregistered}$ son nombres
registrados por MathWorks\\


Versión 0.10

Pendientes de ampliación las secciones marcadas con (+)

Typeset by \LaTeX{}

Escrito en Kubuntu GNU/Linux y Gentoo GNU/Linux. No ha sido necesaria
ninguna herramienta comercial para preparar este texto, sirva como
demostración que el software no por ser más caro debe ser mejor.

pdf en \url{http://torroja.dmt.upm.es:9673/Guillem_Site/CursoMatlab/cursomo.pdf}


\chapter*{Notas y agradecimientos}
\subsection*{Versión 0.10}
Las puntualizaciones para esta versión son más una fe de erratas que
un conjunto de aclaraciones.  Marian me puntualizó que en algunos
casos confundía el concepto de rango con el de dimensión de una
matriz.  Esta confusión está diseminada por todo el libro y de momento
no tengo tiempo para releerlo entero.  Es un error que por conocido no
deja de ser grave.

Otro error que debe ser subsanado es la afirmación, sobre todo en los
apéndices, de que Matlab no tiene ningún método de optimización
automática y que todos los bucles son lentos.  Mientras esto sigue
siendo cierto para Octave, Matlab cuenta con dos herramientas muy
interesantes; son el \emph{profiler} y el optimizador JIT (Just In
Time).  Con un uso inteligente de la optimización automática la
velocidad de los códigos simples se acerca a la obtenida con un
lenguaje compilado.

Sin embargo no deja de ser cierto que la mejor manera de acelerar
código en Matlab es vectorizar y que obligar al usuario a optimizar
manualmente en vez de generar bytecode a partir de las funciones no es
camino a seguir.  En eso Python sigue siendo un ejemplo de mejor
diseño.


\tableofcontents{}

\listoffigures

%%%%%%%%%%%%%%%%%%%%%%%%%%%%%%%%%%%%%%%%%%%%%%%%%%

\part{Introducción y elementos del lenguaje Matlab}

%introducción
\input{partI/intro.tex}
%Elementos del lenguaje Matlab
\input{partI/matlab.tex}

%%Parte II, La biblioteca de funciones

\part{La biblioteca de funciones}

%Matrices y álgebra lineal
\input{partfunc/matalg.tex}
%Gráficos
\input{partfunc/graficos.tex}
%Cálculo y análisis
\input{partfunc/calculo.tex}
%toolkits
\input{partfunc/toolkits.tex}
%temas avanzados
\input{partfunc/avanzado.tex}


%%Parte II, ejercicios

\part{Ejercicios}

%Ejercicios resueltos
\input{partejer/resueltos.tex}
%Ejercicios propuestos
\input{partejer/propuestos.tex}

\appendix

\part{Apéndices}
%Los diez mandamientos de la programación en Matlab
\input{apendices/diezmand.tex}
%Pequeña introducción a LaTeX
\input{apendices/latex.tex}
%Software y formatos libres
\input{apendices/freesoft.tex}
%Lo que no me gusta de Matlab
\input{apendices/matlabkk.tex}
%Licencia creative commons



\begin{thebibliography}{99}

\bibitem{Turbulence}

  Turbulence and Vortex Dynamics; Javier Jiménez; July 26, 2004;
  http://torroja.dmt.upm.es/

\bibitem{Fortran}

  Fortran 95/2003 Explained; Michael Metcalf, John Reid, Malcolm
  Cohen; Oxford University Press; 2004.

\bibitem{Numerical}

  Numerical Recipes in Fortran 77, The Art of Scientific Computing;
  Second Edition; William H. Press, Saul A. Teukolsky, William T.
  Vetterling, Brian P. Flannery; Cambridge University Press, 1997

\bibitem{Orlandi}

  Fluid Flow Phenomena, A Numerical Toolkit; Paolo Orlandi; Kluwer
  Academic Publishers, 2000.

\bibitem{Chung}

  Computational Fluid Dynamics; T.S. Chung; Cambridge University
  Press, 2002

\bibitem{Schildt}

  C++: The Complete Reference; Herbert Schildt; Tata Mc Graw-Hill
  ,1999

\bibitem{OctaveWiki}

  Scary Octave; http://wiki.octave.org/

\bibitem{OctaveManual}

  The Octave Language for Numerical Computations; John W. Eaton;
  http://www.octave.org/

\bibitem{Python}

  Python Documentation; Guido van Rossum; http://www.python.org/doc/

\bibitem{Wolfram}

  MathWorld; Eric W. Weisstein; http://mathworld.wolfram.com/

\bibitem{Fluidos}

  Mecánica de Fluidos; Amable Liñán; Publicaciones ETSIA, Universidad
  Politécnica de Madrid.

\bibitem{Numerico1}

  Cálculo Numérico I; Varios; Publicaciones ETSIA, Universidad
  Politécnica de Madrid.

\bibitem{EDO}

  Análisis y Cálculo Numérico en Ecuaciones en Derivadas Parciales;
  Juan A. Hernández, Eusebio Valero; Publicaciones ETSIA, Universidad
  Politécnica de Madrid.

\bibitem{Trefethen}

  Spectral methods in MATLAB; Lloyd N. Trefethen; SIAM, 2000

\bibitem{Algebra}

  Algebra Lineal y Geometría Cartesiana; Juan de Burgos; McGraw Hill,
  2000

\bibitem{Multicapa}

  Fortran 95: Programación Multicapa para la simulación de sistemas
  físicos; Juan A. Hernández, Mario A. Zamécnik; ADI, 2001.

\bibitem{Wikipedia}

  Wikipedia, the free encyclopedia:\url{http://en.wikipedia.org/wiki/Wikipedia}

\end{thebibliography}





\printindex{}

\end{document}
